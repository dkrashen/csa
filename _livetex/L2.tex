\section{Lecture (1/16)}
Today we will discuss tensors and centralizers. 
\subsection{Tensor Products}
Let $R,S,T$ be rings. Let ${}_RM_S,{}_SN_T$ bi-module, and a map to ${}_RP_T$
$$\phi: M \times N \lrra P$$
We say that $\phi$ is $R-S-T$ linear if
\begin{enumerate}
\item for all $n\in N$, $m\mapsto \phi (m,n)$ is left $R$-module homomorphism;
\item for all $m\in N$, $n\mapsto \phi (m,n)$ is right $T$-module homomorphism;
\item $\phi (ns,m) = \phi(n,sm)$.
\end{enumerate}
\begin{defn}\label{2.2}
Given ${}_RM_S,{}_SN_T$, we say that a bi-module ${}_RP_T$ together with a $R-S-T$ linear map $M\times N \lrra P$ is a \textbf{tensor product} of $M \aad N$ over $S$ is for all $M\times N \lrra Q$ $R-S-T$ linear there exists a unique factorization:
$$
\begin{tikzcd}
M \times N \arrow{r} \arrow{d} & Q \\
P \arrow[dashed]{ru}{\exists !} & {}
\end{tikzcd}$$

\end{defn}
\begin{defn}\label{2.3}
We define $M \otimes_S N$ to be the quotient of the free Abelian group generated by $M \times N$ by the subgroup generated by the relations
\begin{eqnarray*}
(m,n_1 + n_2) & = & (m,n_1) + (m,n_2) \\
(m_1 + m_2,n) & = & (m_1,n) + (m_2,n) \\
(ms,n) & = & (n,sn)
\end{eqnarray*}
\end{defn}
In the case where $R$ commutative, left modules have right module structure and vice versa. In this way, $M_R \otimes_R {}_RN$ has an $R$-modules structure; so when $R$ commutative, we will refer to a $R-R-R$ linear map as $R$ bi-linear. We have the notation that the ordered pair $(m,n)$ is the equivalence class $m\otimes n$, which are called \textbf{simple tensors}. We note that elements in $M\otimes_R N$ are linear combinations of simple tensors.

In the case of tensors over fields, a lot of the structure is much more transparent and simpler.
\begin{prop}\label{2.4}
If $V,W$ are vector space over a field $F$ with bases $\brk{v_i},\brk{w_j}$, then $V\otimes W$ is a vector space with basis given by $\brk{v_i \otimes w_j}.$
\end{prop}
\begin{proof}
Clearly, this basis spans. To see independence, define a function $\phi_{k,l}:V\times W \lrra F$ which maps $(\sum \alpha_iv_i,\sum \beta_j w_j) \longmapsto \alpha_k \beta_l$. This map is bi-linear, and the induced map on tensors is a group homomorphism. Hence we have linear independence.
\end{proof}

If $V/F$ is some vector space $L/F$ field extension, then $L\otimes_F V$ is an $L$-vector space with basis $\brk{1\otimes v_i}$ where $\brk{v_i}$ is a basis for $V$. Similarly, given a linear transformation $T:V \lrra W$, then 
$$L\otimes T : L \otimes V \lrra L\otimes W$$
where $L\otimes T(x\otimes v) \mapsto x \otimes T(v).$ If we identify the bases of $V \aad L \otimes V$, we see that $T$ and $L\otimes T$ have the ``same" matrix. Thus 
$$L\otimes (\ker T) = \ker (L\otimes T),$$
and similarly, for cokernel, image, etc. 

\subsection{Tensor Products of Algebras}
If $A,B$ are $F$-algebras, then $A\otimes B$ is naturally an $F$-algebra since
$$(a\otimes b)(a'\otimes b') = (aa' \otimes bb')
$$
Note that $A,B$ are not necessarily commutative rings, so we are somewhat forcing this construction. In fact, something funny is actually happening. Inside $A\otimes B$, $A\otimes 1 \aad 1\otimes B$ are sub-algebras that are isomorphic to $A \aad B$, respectively. In particular, $A\otimes 1$ commutes with $1\otimes B$.
\begin{prop}\label{2.6}
Suppose $A,B$ are $F$-algebras, then for any $F$-algebra $C$, there is a bijection between the following two sets:
$$\brk{\Hom(A\otimes B,C)} \leftrightarrow \brk{A \rra C,B\rra C \text{ such that images of $A \aad B$ commute in $C$}}$$
\end{prop}
\begin{proof}
The inclusion $\subseteq$ is clear by our previous comment. For the reverse inclusion, $A\otimes B$ is generated as an algebra by $A \otimes 1 \aad 1 \otimes B$. So given $\phi_1:A \lrra C,\phi_2: B \lrra C$, then $\rho : A \otimes B \lrra C$ is defined by $a \otimes b \mapsto \phi_1(a)\cdot \phi_2(b).$
\end{proof}

Given $A,B$ $F$-algebras and ${}_AM_B$ we have homomorphisms $A \lrra \End_F(M)$ and $B^{\op} \lrra \End_F(M).$ Moreover, there images commute i.e., the images of $A,B^{\op}$ commute so $(am)b = a(mb)$. So we get a map 
$$A \otimes B^{\op} \lrra \End_F(M)$$
which defined a left $A\otimes B^{\op}$-modules structure on $M$. Thus, we have a natural equivalence of the categories $A-B$ bi-modules and left $A\otimes B^{\op}$-modules.

\subsection{Commutators}
Given $A/F$ some algebra, and $\Lambda \subset A$, then 
$$C_A(\Lambda) = \brk{a \in A : a \lambda = \lambda a \, \forall a \in A},$$
and $C_A(A) = Z(A)$. Suppose that $M$ is a right $A$-module, then we have a homomorphism $A^{\op} \lrra \End_F(M)$. If we let $C = C_{\End_F(M)}(A^{\op}) = \End_A(M)$. To preserve our sanity, we will regard $M$ as a left $C$-module. This gives $M$ the structure of a $C-A$ bi-module. 

\begin{theorem}[Double Centralizer Theorem Warm-Up]\label{2.8}
Let $B$ be an $F$-algebra, $M$ a faithful, semi-simple right $B$-module, finitely dimensional over $F$. Let $E = \End_F(M)$, $C = C_E(B^{\op})$, then $B^{\op} = C_E(C) = C_E(C_E(B^{\op}))$.
\end{theorem}
\begin{proof}
Let $\phi \in C_E(C)$. Choose $\brk{m_1,\dots , m_n}$ a basis for $M/F$. Write $N = \bigoplus^n M \ni w = (m_1,\dots , m_n)$. Since $M$ is semi-simple, so $N$ is semi-simple. This allows us to write
$$N = wB \oplus N' \fs N'$$
Set $\pi : N\lrra N'$ be a projection (right $B$-module map) that factors through $wB$. Since $\pi \in \End_B(N) = \Mat_n(\End_B(M)) = \Mat_n(C_{\End_F(M)}(B^{\op})) = \Mat_n(C)$. 

Set $\phi^{\oplus n}: N \lrra N$ doing $\phi$ on each entry. Then $w\phi^{\oplus n} = (\pi w)\phi^{\oplus n} = \pi (w\phi^{\oplus n}) = \pi(wb) \in wB$. The general principle is the following:
$\Mat_N(\brk{\cdot})$ commute with ``scalar matrices" whose entries commute with $\brk{\cdot}$, which is why we can move the $w$ inside
\end{proof}
Our next goal is to prove that:
\begin{theorem}\label{2.9}
If $A$ is a CSA/$F$, then $A \otimes_F A^{\op} \iso \End_F(A)$.
\end{theorem}
\begin{proof}
Notice that $A$ is an $A-A$ b-module, so it defines a map $A\otimes A^{\op} \lrra \End_F(A)$. The question is why is this bijective. Suppose that $\brk{a_i}$ is a baiss for $A$ and ($A^{\op}$). We wan to see when 
$$\sum c_{i,j}a_i \otimes a_j \overset{?}{\longmapsto} 0 \in \End (A)$$
More abstractly, if we have $A,B$ commuting sub-algebras of $E$. Let $a_i \in A$, $b_j \in B$ be linearly independent over $F$, then $a_ib_j$ is independent in $E$. Since $E$ is an $A-A$ bi-module, so $A \otimes A^{\op}$ left module. $E$ is also a right $B$-module, in particular $A\otimes A^{\op} - B$ bi-module. $A$ is a CSA, so it is a simple $A\otimes A^{\op}$-module, and $\End_{A\otimes A^{\op}}(A) = F = Z(A).$ Thus 
$$C_{\End_F(A)}(C_{\End_F(A)}(\im(A\otimes A^{\op}))) = C_{\End_F(A)}(F) = \End_F(A).$$
Then Theorem \ref{2.8} tells us that $\im (A\otimes A^{\op}) = \End_F(A),$ which is what we desired.\footnote{There was a lot of confusion on this proof. Review Danny's online notes for valid proof.} 
\end{proof}

Thus, if $A$ is a CSA, then $A \otimes A^{\op} \iso \End_F(A) = \Mat_n(F)$, where $n = \dim_F(A)$. 

\begin{prop}\label{2.10}
$A$ is a CSA if and only if there exists $B$ such that $A \otimes B \iso \Mat_n(F).$
\end{prop}
\begin{proof}
$(\Rightarrow).$ This is clear. $(\Leftarrow).$ If $A \otimes B \iso M_n(F)$, note that $M_n(F)$ are central simple. If $I \leq A$, then $I \otimes B \leq \Mat_n(F)$ by dimension counting. If $I$ is non-trivial, so is $I \otimes B$, hence $A $ is simple. Thus, $Z(A) = C_{\Mat_n(F)}(A) \cap A$. We know that $B \subset C_{\Mat_n(F)(A)}$, which implies that $A \otimes C_{\Mat_n(F)}(A) \ext \Mat_n(F).$ But we also know that $A\otimes B \iso \Mat_n(F)$ by assumption, hence we have $B = C_{\Mat_n(F)}(A)$. Thus $Z(A) = C_{\Mat_n(F)}(A) \cap A = B \cap A = F$. 
\end{proof}

\begin{prop}\label{2.11}
$A$ is a CSA/$F$ if and only if for all field extensions $L/F$ such that $L \otimes_F A$ CSA/$L$ if and only if $\overline{F}\otimes_F A \iso \Mat_n(\overline{F}).$
\end{prop}

\begin{proof}
$A$ is a CSA $\Rightarrow A \otimes A^{\op} \iso \Mat_n(F) \Rightarrow (A \otimes_F A^{\op}) \otimes_F L \iso \Mat_n(L).$ Notice that we can re-write $(A \otimes_F A^{\op}) \otimes_F L = (A\otimes L)\otimes_L (A^{\op} \otimes L)$, so by Proposition \ref{2.10}, we have that $A\otimes L$ is a CSA for all $L$. In particular, $A\otimes_F \overline{F}$ is a CSA. Thus by Theorem \ref{1.27}, $A\otimes_F \overline{F} \iso \Mat_n(D)$ for some finite dimensional division algebra $D/\overline{F}$. Hence for all $d \in D\unit$, $\overline{F}[d]/\overline{F}$ is a finite extension of $\overline{F}$. Since it is a finite extension, $d \in \overline{F}$, which implies that $D = \overline{F}$ i.e., $A \otimes_F \overline{F} \iso \Mat_n(\overline{F}).$

Now suppose that $A \otimes_F \overline{F} \iso \Mat_n(\overline{F}).$ So $A$ must be simple, otherwise, $I \otimes \overline{F} \leq A \otimes \overline{F} = \Mat_n(\overline{F}).$ Now we want to show that $Z(A \otimes \overline{F}) = Z(A) \otimes \overline{F}$. This is true by considering the kernel of a linear map and just extending scalars.  
\end{proof}

\begin{defn}\label{2.12}
If $A$ is a CSA, then $\deg A = \sqrt{\dim_F(A)}$. This makes sense since $\overline{F} \otimes A \iso \Mat_n(\overline{F})$ has dimension $n^2$. 
\end{defn}
\begin{defn}\label{2.13}
By Theorem \ref{1.27}, $A \iso \Mat_n(D)$, and we can check that $Z(D) = F$, hence $D$ is a CSA, which we will call a \textbf{central division algebra (CDA)}. We define the \textbf{index of A} as $\ind (A) = \deg (D)$, where $D$ is the underlying division algebra. We know that this is unique up to isomorphism, since $D = \End_A(P)$ , where $P$ is a simple right $A$-module.
\end{defn}
\begin{remark}\label{2.14}
Note that 
$$\dim_F(A) = m^2 \dim_F (D)$$
so that $\deg A = m \deg D = m \ind A$, and in particular, $\ind A | \deg A$.
\end{remark}
\subsection{Brauer Equivalence}
\begin{defn}\label{2.16}
CSA's $A,B$ are \textbf{Brauer equivalent} $A \eqr B$ if and only if there exists $r,s$ such that $\Mat_r(A) \iso \Mat_s(B)$. This essentially says that $\Mat_r(\Mat_n(D_A)) \iso \Mat_s(\Mat_m(D_B))$ , which implies that $D_A \iso D_B$. Alternatively, 
$$A \eqr B \lifaf \text{underlying divison algebras are isomorphic.}$$
\end{defn}

\begin{fact}
If $A,B/F$ are CSA's, then $A \otimes_F B$ is also a CSA. The ``cheap" way to prove this is to just tensor over $\overline{F}$ and see what happens.
\end{fact}
\begin{defn}\label{2.17}
The \textbf{Brauer group} $\Br (F)$ is the group of Brauer equivalence classes of CSA's over $F$ with operation $[A] + [B] = [A\otimes_F B].$ The identity element is $[F]$, and note that 
$$[A] + [A^{\op}] = [A \otimes_F A^{\op}] = [\Mat_{\dim_FA}(F)] = [F].$$
\end{defn}

\begin{defn}\label{2.18}
The \textbf{exponent of A} (or \textbf{period of A}) is the order of $[A]$ in $\Br (F).$
\end{defn}

\begin{fact}
We will show that $\per A | \ind A$.
\end{fact}

